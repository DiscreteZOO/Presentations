\documentclass[usepdftitle=false]{beamer}
\usepackage[slovene,english]{babel}
\selectlanguage{english}

\setbeamertemplate{navigation symbols}{}
\usepackage{beamerthemedefault}
\beamertemplatefootempty
\usepackage[utf8]{inputenc}
\usepackage{amssymb}
\usepackage{amsmath}
\usepackage{mathdots}
\usepackage{color}
\usepackage{enumerate}
\usepackage{graphicx,epsfig}
\usepackage{tikz,pgflibraryshapes}

\usetikzlibrary{calc}
\usetikzlibrary{backgrounds}

\hypersetup{pdftitle={DiscreteZOO}}

% za generiranje slik, da se potem hitreje prevaja, pozeni
% pdflatex --jobname=imeSlike morelia-zoo.tex
% kjer je imeSlike doloceno z ukazom \beginpgfgraphicnamed (pgfmanual p. 500)
\pgfrealjobname{morelia-zoo}

\definecolor{zooteal}{rgb}{0.01,0.44,0.5}
\definecolor{zoogray}{gray}{0.97}
\definecolor{zoored}{rgb}{0.94,0.43,0.29}
\setbeamercolor*{title}{fg=zooteal}
\setbeamercolor*{subtitle}{fg=zooteal}
\setbeamercolor*{palette primary}{fg=zoored,bg=zoogray}
\setbeamercolor*{structure}{fg=zooteal,bg=zoogray}

\title[DiscreteZOO]{\includegraphics[height=2cm]{discretezoo.png}}
\subtitle{A repository of graphs and other discrete objects}
\author{Janoš Vidali \\
{\scriptsize Joint work with Katja Berčič}
}
\date{January 25--26, 2018}

\newcommand{\set}[2]{\ensuremath{\left\{ #1 \; \middle| \; #2 \right\}}}
\newcommand{\Z}{\ensuremath{\mathbb{Z}}}
\newcommand{\N}{\ensuremath{\mathbb{N}}}
\newcommand{\R}{\ensuremath{\mathbb{R}}}
\newcommand{\F}{\ensuremath{\mathbb{F}}}
\newcommand{\M}{\ensuremath{\mathcal{M}}}
\newcommand{\e}{\ensuremath{{\textbf e}}}
\newcommand{\PG}{\ensuremath{\operatorname{PG}}}
\newcommand{\GQ}{\ensuremath{\operatorname{GQ}}}
\newcommand{\Alt}{\ensuremath{\operatorname{Alt}}}
\newcommand{\Her}{\ensuremath{\operatorname{Her}}}
\newcommand{\rank}{\ensuremath{\operatorname{rank}}}
\newcommand{\dbab}{\ensuremath{(d, b, \alpha, \beta)}}
\newcommand{\pp}{\ensuremath{\!+\!}}
\newcommand{\mm}{\ensuremath{\!-\!}}

\newcommand{\Wlog}{WLOG }
\newcommand{\theor}{{\bf Theorem}: }
\newcommand{\lem}{{\bf Lemma}: }
\newcommand{\prop}{{\bf Proposition}: }
\newcommand{\conj}{{\bf Conjecture}: }
\newcommand{\prob}{{\bf Problem}: }

\newcommand{\m}{\ensuremath{\!-\!}}
\newcommand{\p}{\ensuremath{\!+\!}}

\newcommand{\sV}[2]{{
       \setlength{\arraycolsep}{2pt}
       \renewcommand{\arraystretch}{0.6}
       \left[\begin{array}{ccc} #1 \\ #2 \end{array}\right]
}}

\newcommand{\sVa}[2]{{
       \setlength{\arraycolsep}{2pt}
       \renewcommand{\arraystretch}{0.6}
       \left[\begin{array}{ccc} #2 \end{array}\right]
}}
% odkomentiraj za velike simbole v tabelah
%\renewcommand{\sVa}[2]{\sV{#1}{#2}}

\newcommand{\sS}[2]{{
	\setlength{\arraycolsep}{2pt}
	\renewcommand{\arraystretch}{0.6}
	\left\{\begin{array}{ccc} #1 \\ #2 \end{array}\right\}
}}

\definecolor{darkgreen}{rgb}{0.0,0.5,0.0}

\newcommand{\res}[1]{\textcolor{darkgreen}{#1}}
\newcommand{\defn}[1]{\textcolor{zoored}{\emph{#1}}}
\newcommand{\keyw}[1]{\textcolor{zooteal}{#1}}
\newcommand{\con}[1]{\textcolor{zoored}{#1}}

\begin{document}

{
\section{Introduction}
\frame[plain]{
\titlepage
}
}

\frame
{
    \frametitle{Idea}

    \begin{itemize}
    \itemsep=5mm
    \item There are various \keyw{censuses} of graphs and \\
    other discrete objects in the literature and the internet.
    \item We want these objects to be \keyw{easily accessible}.
    \item Furthermore, we want to have \keyw{precomputed properties} \\
    that can be used for \keyw{searching}.
    \end{itemize}
}

\frame
{
    \frametitle{DiscreteZOO}

    \begin{itemize}
    \itemsep=5mm
    \item We gather the censuses into a \keyw{database}.
    \item The data can be accessed via the website
    \begin{center}
    \vskip 0.2cm
    \Large \con{\href{http://discretezoo.xyz}{discretezoo.xyz}}
    \end{center}
    \item A \keyw{Sage} package is also available at
    \begin{center}
    \vskip 0.2cm
    \Large \con{\href{https://github.com/DiscreteZOO}{github.com/DiscreteZOO}}
    \end{center}
    \end{itemize}
}

\frame
{
    \frametitle{Where we are now}

    \begin{itemize}
    \itemsep=5mm
    \item \con{Censuses}
        \begin{itemize}
        \itemsep=3mm
        \item All connected \keyw{cubic vertex-transitive} graphs with \\
        at most \defn{$1280$} vertices (by P.~Potočnik, P.~Spiga and G.~Verret),
        \item all connected \keyw{cubic arc-transitive} graphs with at most
        \defn{$2048$} vertices (from the extended Foster census by M.~Conder),
        and
        \item all \keyw{vertex-transitive graphs} with at most \defn{$31$}
        vertices \\ (by G.~Royle).
        \end{itemize}
    \item \con{Computed properties}
        \begin{itemize}
        \itemsep=3mm
        \item Basic graph properties (\keyw{order}, \keyw{degree},
        \keyw{diameter}, \keyw{girth}, \dots),
        \item automorphism group related properties \\
        (\keyw{vertex-}, \keyw{edge-}, \keyw{arc-},
        \keyw{distance-transitivity}),
        \item some other properties \\
        (\keyw{is Hamiltonian}, \keyw{is Cayley}, \keyw{is partial cube},
        \dots).
        \end{itemize}
    \end{itemize}
}

\frame
{
    \frametitle{Where we are not quite yet}
    (but we are working on it!)
    \vskip 1cm

    \begin{itemize}
    \itemsep=5mm
    \item More graphs
    \item More precomputed properties
    \item Nice \keyw{images} of graphs
    \item Other combinatorial objects
        \begin{itemize}
        \itemsep=3mm
        \item \keyw{Finite groups}, \keyw{polytopes}, \keyw{maniplexes},
        \keyw{geometries}, \dots
        \item Feasible \keyw{parameter sets} for objects \\
        (e.g.~distance-regular graphs)
        \end{itemize}
    \item \con{Your wishes?}
    \end{itemize}
    \vskip 1cm
}

\frame
{
\begin{center}
\huge
\con{Demo}
\end{center}
}


\frame
{
\begin{center}
\huge
\con{Questions, ideas?}
\end{center}
}

\end{document}
